\documentclass[11pt]{exam}
\usepackage{graphicx, bm, listings}
\title{Computational Logic - Assignment 2}
\author{Alessandro Marostica}
\date{\today}
\begin{document}
\maketitle
\section*{Exercise 2}
A bijective function is both injective and surjective.
We define a function \(f: D \rightarrow C\). We then express its injectivity and surjectivity respectively as such:
\\
\begin{center}
    \(\forall x_1 \in D \  \forall x_2 \in D((x_1 \neq x_2) \Longrightarrow f(x_1) \neq f(x_2)) \)
    \\
    \(\forall y \in C \  \exists x \in D(y = f(x))\)
\end{center}
We can then introduce conjunction between the two formulas to descirbe a bijective funciton:
\begin{center}
    \(\forall x_1 \in D \  \forall x_2 \in D((x_1 \neq x_2) \Longrightarrow f(x_1) \neq f(x_2)) \  \forall y \in C \  \exists x \in D(y = f(x))\)
\end{center}
\end{document}