\documentclass[11pt]{exam}
\usepackage{graphicx, bm}
\title{Computational Logic - Assignment 1}
\author{Alessandro Marostica}
\date{\today}
\begin{document}
\maketitle
\section*{Prelude}
I have never used LaTeX before, please bear with my plausibly horrible usage.
\section*{Exercise 2}
\paragraph*{Disjunction}
Using De Morgan's law, one can define the OR operator as such:
\begin{center}
    \(A \vee B \dashv \vdash \neg(\neg A \wedge \neg B)\)
\end{center}
\begin{center}
    \begin{tabular}{|c|c|c|}
        \hline
        \(A\) & \(B\) & \(A \vee B\) \\
        \hline
        T & T & T \\
        T & F & T \\
        F & T & T \\
        F & F & F \\
        \hline
    \end{tabular}
    \begin{tabular}{|c|c|c|c|c|c|}
        \hline
        \(A\) & \(B\) & \(\neg A\) & \(\neg B\) & \(\neg A \wedge \neg B\) & \(\neg(\neg A \wedge \neg B)\) \\
        \hline
        T & T & F & F & F & T \\
        T & F & F & T & F & T \\
        F & T & T & F & F & T \\
        F & F & T & T & T & F \\
        \hline       
    \end{tabular}
\end{center}
\paragraph*{Implication} 
One can define IMPLIES as such:
\begin{center}
    \(A \rightarrow B \dashv \vdash \neg A \vee B\)
\end{center}
\begin{center}
    \begin{tabular}{|c|c|c|}
        \hline
        \(A\) & \(B\) & \(A \rightarrow B\) \\
        \hline
        T & T & T \\
        T & F & F \\
        F & T & T \\
        F & F & T \\
        \hline    
    \end{tabular}
    \begin{tabular}{|c|c|c|c|}
        \hline
        \(A\) & \(B\) & \(\neg A\) & \(\neg A \vee B\) \\
        \hline
        T & T & F & T \\
        T & F & F & F \\
        F & T & T & T \\
        F & F & T & T \\
        \hline       
    \end{tabular}
\end{center}
\paragraph*{Material equivalence}
One can define IFF as such:
\begin{center}
    \(A \dashv \vdash B \dashv \vdash (A \wedge B)\vee(\neg A \wedge \neg B)\)
\end{center}
\begin{center}
    \begin{tabular}{|c|c|c|}
        \hline
        \(A\) & \(B\) & \(A \dashv \vdash B\) \\
        \hline
        T & T & T \\
        T & F & F \\
        F & T & F \\
        F & F & T \\
        \hline
    \end{tabular}
    \begin{tabular}{|c|c|c|c|c|c|c|}
        \hline
        \(A\) & \(B\) & \(A \wedge B\) & \(\neg A\) & \(\neg B\) & \(\neg A \wedge \neg B\) & \((A \wedge B)\vee(\neg A \wedge \neg B)\) \\
        \hline
        T & T & T & F & F & F & T \\
        T & F & F & F & T & F & F \\
        F & T & F & T & F & F & F \\
        F & F & F & T & T & T & T \\
        \hline
    \end{tabular} 
\end{center}
Note that IFF yields the same result as XNOR, which is opposite of XOR.
\paragraph*{Exclusive disjunction}
One can define XOR as such:
\begin{center}
    \(A \oplus B \dashv \vdash (A \wedge \neg B)\vee(\neg A \wedge B)\)
\end{center}
\begin{center}
    \begin{tabular}{|c|c|c|}
        \hline
        \(A\) & \(B\) & \(A \oplus B\) \\
        \hline
        T & T & F \\
        T & F & T \\
        F & T & T \\
        F & F & F \\
        \hline
    \end{tabular}
    \begin{tabular}{|c|c|c|c|c|c|c|}
        \hline
        \(A\) & \(B\) & \(\neg A\) & \(\neg B\) & \(A \wedge \neg B\) & \(\neg A \wedge B\) & \((A \wedge \neg B)\vee(\neg A \wedge B)\) \\
        \hline
        T & T & F & F & F & F & F \\
        T & F & F & T & T & F & T \\
        F & T & T & F & F & T & T \\
        F & F & T & T & F & F & F \\
        \hline
    \end{tabular}
\end{center}
\paragraph*{Defining simple operators with NAND} \hfill \break
NAND is defined as such:
\begin{center}
    \(p \uparrow q \dashv \vdash \neg(p \wedge q)\)
\end{center}
hence:
\subparagraph*{\(\bm{\neg p}\)}
can be defined as \(p \uparrow p\) (\(\uparrow\) is the symbol for NAND), this is equivalent to \(p \uparrow p \dashv \vdash \neg(p \wedge p) \dashv \vdash \neg p\). In natural language this can be referred to as "not both p and p".
\subparagraph*{\(\bm{p \wedge q}\)}
can be defined as \(\neg(p \uparrow q)\), this is equivalent to \(\neg(p \uparrow q) \dashv \vdash \neg(\neg(p \wedge q)) \dashv \vdash p \wedge q\)
\subparagraph*{\(\bm{p \vee q}\)}
can be defined as \((p \uparrow p) \uparrow (q \uparrow q)\), this is equivalent to \((p \uparrow p) \uparrow (q \uparrow q) \dashv \vdash \neg(p \wedge p) \uparrow \neg(q \wedge q) \dashv \vdash \neg p \uparrow \neg q \dashv \vdash \neg (\neg p \wedge \neg q) \dashv \vdash p \vee q\). In the last step De Morgan's law is applied.
\subparagraph*{\(\bm{p \rightarrow q}\)}
can be defined as \(p \uparrow (q \uparrow q)\), this is equivalent to \(p \uparrow (q \uparrow q) \dashv \vdash p \uparrow \neg q \dashv \vdash \neg p \vee \neg \neg q \dashv \vdash \neg p \vee q \dashv \vdash p \rightarrow q\). In this proof the rules of disjunction in terms of NAND and negation in terms of NAND are applied.
\subparagraph*{\(\bm{p \leftrightarrow q}\)}
can be defined as \(((p \uparrow p) \uparrow (q \uparrow q)) \uparrow (p \uparrow q)\), this is equivalent to \(((p \uparrow p) \uparrow (q \uparrow q)) \uparrow (p \uparrow q) \dashv \vdash \neg(((p \uparrow p) \uparrow (q \uparrow q)) \wedge (p \uparrow q)) \dashv \vdash \neg((p \vee q) \wedge (p \uparrow q)) \dashv \vdash \neg((p \vee q) \wedge \neg (p \wedge q)) \dashv \vdash \neg(p \oplus q) \dashv \vdash p \leftrightarrow q\). In this proof many rules are applied: disjunction interms of NAND, definition of XOR, XOR is negation of biconditional.
\end{document}