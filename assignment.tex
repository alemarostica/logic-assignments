\documentclass[11pt]{article}
\usepackage{graphicx, bm}
\title{Computational Logic - Assignment 1}
\author{Alessandro Marostica\thanks{Arianna, per avermi stimolato ad usare LaTeX}}
\date{\today}
\begin{document}
\maketitle
\section*{Prelude}
I have never used LaTeX before, please bear with my plausibly horrible usage.
\section*{Exercise 2}
\paragraph*{Disjunction}
Using De Morgan's law, one can define the OR operator as such:
\begin{center}
    \(A \vee B \equiv \neg(\neg A \wedge \neg B)\)
\end{center}
\begin{center}
    \begin{tabular}{|c|c|c|}
        \hline
        \(A\) & \(B\) & \(A \vee B\) \\
        \hline
        T & T & T \\
        T & F & T \\
        F & T & T \\
        F & F & F \\
        \hline
    \end{tabular}
    \begin{tabular}{|c|c|c|c|c|c|}
        \hline
        \(A\) & \(B\) & \(\neg A\) & \(\neg B\) & \(\neg A \wedge \neg B\) & \(\neg(\neg A \wedge \neg B)\) \\
        \hline
        T & T & F & F & F & T \\
        T & F & F & T & F & T \\
        F & T & T & F & F & T \\
        F & F & T & T & T & F \\
        \hline       
    \end{tabular}
\end{center}
\paragraph*{Implication} 
One can define IMPLIES as such:
\begin{center}
    \(A \rightarrow B \equiv \neg A \vee B\)
\end{center}
\begin{center}
    \begin{tabular}{|c|c|c|}
        \hline
        \(A\) & \(B\) & \(A \rightarrow B\) \\
        \hline
        T & T & T \\
        T & F & F \\
        F & T & T \\
        F & F & T \\
        \hline    
    \end{tabular}
    \begin{tabular}{|c|c|c|c|}
        \hline
        \(A\) & \(B\) & \(\neg A\) & \(\neg A \vee B\) \\
        \hline
        T & T & F & T \\
        T & F & F & F \\
        F & T & T & T \\
        F & F & T & T \\
        \hline       
    \end{tabular}
\end{center}
\paragraph*{Material equivalence}
One can define IFF as such:
\begin{center}
    \(A \leftrightarrow B \equiv (A \wedge B)\vee(\neg A \wedge \neg B)\)
\end{center}
\begin{center}
    \begin{tabular}{|c|c|c|}
        \hline
        \(A\) & \(B\) & \(A \leftrightarrow B\) \\
        \hline
        T & T & T \\
        T & F & F \\
        F & T & F \\
        F & F & T \\
        \hline
    \end{tabular}
    \begin{tabular}{|c|c|c|c|c|c|c|}
        \hline
        \(A\) & \(B\) & \(A \wedge B\) & \(\neg A\) & \(\neg B\) & \(\neg A \wedge \neg B\) & \((A \wedge B)\vee(\neg A \wedge \neg B)\) \\
        \hline
        T & T & T & F & F & F & T \\
        T & F & F & F & T & F & F \\
        F & T & F & T & F & F & F \\
        F & F & F & T & T & T & T \\
        \hline
    \end{tabular} 
\end{center}
Note that IFF yields the same result as XNOR, which is opposite of XOR.
\paragraph*{Exclusive disjunction}
One can define XOR as such:
\begin{center}
    \(A \oplus B \equiv (A \wedge \neg B)\vee(\neg A \wedge B)\)
\end{center}
\begin{center}
    \begin{tabular}{|c|c|c|}
        \hline
        \(A\) & \(B\) & \(A \oplus B\) \\
        \hline
        T & T & F \\
        T & F & T \\
        F & T & T \\
        F & F & F \\
        \hline
    \end{tabular}
    \begin{tabular}{|c|c|c|c|c|c|c|}
        \hline
        \(A\) & \(B\) & \(\neg A\) & \(\neg B\) & \(A \wedge \neg B\) & \(\neg A \wedge B\) & \((A \wedge \neg B)\vee(\neg A \wedge B)\) \\
        \hline
        T & T & F & F & F & F & F \\
        T & F & F & T & T & F & T \\
        F & T & T & F & F & T & T \\
        F & F & T & T & F & F & F \\
        \hline
    \end{tabular}
\end{center}
\paragraph*{NAND is sufficient to define all simple logical operators}
\subparagraph*{\(\bm{\neg p}\)}
can be defined as \(p \uparrow p\) (\(\uparrow\) is the symbol for NAND), this is equivalent to \(p \uparrow p \leftrightarrow \neg(p \wedge p) \leftrightarrow \neg p\). In natural language this can be referred to as "not both p and p".
\subparagraph*{\(\bm{p \wedge q}\)}
can be defined as \(\neg(p \uparrow q)\)
\end{document}